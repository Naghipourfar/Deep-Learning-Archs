\documentclass{article}
\pagestyle{empty}
\usepackage{graphicx}
\usepackage{hyperref}
\usepackage{listings}
\usepackage[margin=2.5cm]{geometry}
\setlength\parindent{0pt}
\usepackage{enumitem}
\usepackage{amsmath}
\usepackage{float}
\usepackage[extrafootnotefeatures]{xepersian}
\settextfont[Scale=1]{B Zar}

\begin{document}
\begin{titlepage}
	\centering
	{\scshape\LARGE به نام خداوند بخشنده و مهربان \par}
	\vspace{2cm}
	{\huge\bfseries یادگیری عمیق\par}
	\vspace{3cm}
	{\Large\bfseries  تمرین پنجم \par}
	\vspace{3cm}
	{\Large\itshape 		محسن نقی‌پورفر		94106757\par}
	\vspace{0.25cm}
	\vfill
	\end{titlepage}
%\begin{figure}[H]
%	\centerline{\includegraphics[width=10cm, height=6cm]{CE1}}
%	\caption{نمودار تابع هزینه آن برای داده‌های یادگیری و تست}
%\end{figure}
\section{\lr{Regularization}}
\subsection{\lr{BatchNormalization}}
\subsubsection{تاثیر اضافه کردن \lr{BatchNormalization}}
\subsubsection{تعداد پارامتر های افزوده شده}
\subsubsection{پیاده سازی تابع این لایه}
\subsection{\lr{Dropout}}
\subsubsection{تاثیر اضافه کردن \lr{Dropout}}
\subsubsection{فرق در آموزش و تست}
\subsubsection{پیاده سازی تابع این لایه}
\section{\lr{Google Colab}}
\subsection{گزارش نتیجه و مراحل اجرا در این محیط}
\subsection{مقایسه در حالت وجود یا عدم وجود منظم ساز‌ها}
\subsection{گزارش نتیجه در اثر وجود دو منظم ساز}

\section{\lr{Visualization}}
\subsection{توضیح در مورد شبکه \lr{VGG}}
\subsection{توضیح معماری شبکه \lr{VGG}}
\subsection{گزارش فیلتر‌های هر لایه و مقایسه اولین و آخرین فیلتر}
\subsection{تحلیل نتایج لایه‌های ۳ و ۱۳ حاصل از ورودی های جدید شبکه}
\section{\lr{DeConvolution}}
\subsection{رسم شبکه عصبی و مشخصات هر لایه}
\subsection{کاربر در شبکه های عمیق}
\subsection{نحوه عملکر این لایه و تفاوت با لایه \lr{Convolution}}


\end{document}